\section{introduction}
\label{sec:intro}

In the Internet of Things, an average person is expected to have as many as five internet enabled devices~\cite{cisco}. Many of these devices will use unmanaged networks such as  802.11x. As the number of devices that use this type of network grows, so does the need to discover efficient  methods to fairly share the network resources. In unmanaged small-scale networks, the performance is dominated
by interference caused by packet collisions as opposed to the loss from background noise ~\cite{jigsaw}. Designing an efficient protocol that can achieve high performance in this scenario is the topic of this paper. 

A large fraction of the collisions in an unmanaged network are caused by hidden terminals.  The Hidden Terminal Problem arises in a network that uses carrier sense multiple access systems (CSMA). For this type of network, at least two terminals  A and B want to communicate with an access point (AP), but do not know of 
the other station and because it cannot sense the other station's transmission. As a result, packets from both terminal A and terminal B arrive at the AP and collide with each other. For standard protocols, both packets are discarded due to this interference and no acknowledgement (ACK)  of successful reception is sent from the AP to each transmitted terminal. After a protocol dependent timeout interval, both terminals retransmit their entire packets again. This process continues, perhaps with a changing retransmission interval, until either the packets are decoded successfully or until the maximum retransmission counter is exceeded. Accordingly, the hidden terminal problem is a severe impairment in unmanaged network due to the multitude of retransmissions.

A large body of research has considered the hidden terminal problem and this research is reviewed in Section \ref{s:related}. This paper presents a new protocol that utilizes a combination of incremental redundancy and opportunistic successive interference cancellation to mitigate the hidden terminal problem. This protocol, called RoXOR, is 802.11 compliant and can increase the network performance whenever hidden terminals are present under the constraint that the network traffic is sparse  and bursty so that the co-channel interference is intermittent. When there is significant interference,  RoXOR has a similar performance to state-of-the art protocols such as Zigzag~\cite{zigzag}.

The key contributions of RoXOR include avoiding explicit feedback by using variable incremental redundancy. In this sense, RoXOR's code is rateless. On the first transmission, the original uncoded data is sent. Only in the event of a packet loss does the sender transmit redundancy in the form of ``check'' bits. Accordingly, RoXOR does not introduce any overhead in the absence of collisions as would be the case if a standard error-control code was used. 
The delay due to retransmission is also reduced because  we will show that the size of the coded, retransmitted packet is smaller than the original uncoded packet.

A unique feature of RoXOR's code is that it explicitly accounts for a priori knowledge about the interference structure instead of assuming random errors throughout the packet, as would be the case if we presumed the bit error in the packet were random. This explicit code structure assumes that the errors occur in contiguous bursts. This enables an efficient decoding based on successive interference cancellation. By exploiting this structure, we can achieve significantly better performance for the same amount of incremental redundancy as compared to a code that does incorporate the structure of the interference in the decoding process.  We simulate our approach using realistic channel models, effectively demonstrating its practicality in an 802.11 environment.

The paper is organized as follows. In Section \ref{s:related}, work related to RoXOR is presented. In Section \ref{sec:htproblem}, we describe the structure of the RoXOR protocol emphasizing how the combination of the incremental 
redundancy and opportunistic successive interference cancellation can efficiently combat collisions caused by hidden terminals. Section \ref{sec:practical} discusses practical issues for RoXOR to be 802.11 compliant while Section 
\ref{sec:impl} discussed our implementation. Section \ref{sec:eval} evaluates the simulation using several benchmarks while Section \ref{sec:conclusion} is a conclusion.