\section{Practical Considerations}
\label{sec:practical}

Modern 802.11 systems use OFDM, we will use BPSK/QPSK for simplicity.

1. OFDM: Segmentation of packet is done depending on the size of the ofdm symbol.
(a) Zigzag cannot be applied to OFDM systems because OFDM symbols are not chunk decodable. 
(b) However, there are solutions to this issue given in [?] and [?] etc.
(c) For simplicity, we use BPSK/QPSK.

2. Applying Folding to Higher Order Modulation (M-QAM)
(a) Folding extends to higher modulations as well.

3. Capture Effect
(a) Capture effect occurs when user A has a much higher SNR than user B; hence, the receiver decodes user A first and treats user B as noise. 
(b) We do not consider the capture effect results.

4. Detecting multiple collisions 
(a) Use the correlation approach proposed by zigzag

5. Channel Estimation
(a) Flat Fading Model (Current Model we are Assuming):

\begin{equation}
\label{eq:flat}
y[n]=H_{1}x_{1}[n]+H_{2}x_{2}[n]+w[n]
\end{equation}

i. slow flat-fading channel, gaussian noise

(b) Frequency Selective Fading Model:

\begin{equation}
\label{eq:selective}
y[n]=h_{1}[n]*x_{1}[n]+h_{2}[n]*x_{2}[n]+w[n]
\end{equation}

i. Equalization is substantially more complicated as compared to flat fading case.
ii. The most recent 802.11 standard uses OFDM.
iii. There are several ways to estimate the channel (i.e. preamble, pilot symbols)

(c) It turns out that the channel (i.e. as measured by the USRP) we observe is flat and slowly varying. 
(d) Since we are using BPSK/QPSK, channel estimation is accomplished by using an AGC and PLL

6. Frequency offset estimation // Fine Channel and Phase Estimation
(a) Costas PLL 
(b) Band-Edge Filter

7. Timing Offset Estimation
(a) Gardner Timing Algorithm
(b) Polyphase Filter Bank Estimation

8. Successive Interference Cancellation
(a) Use the folded linear combinations combined with SiC to recover original packet.

9. Generalization to three or more nodes.
(a) Use a run-length limited type of code, such that there is always a minimum distance between symbols used for ``folding''.
